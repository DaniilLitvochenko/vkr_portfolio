% "ИСР"

\documentclass[a4paper,12pt]{article} % тип документа

% report, book

%  Русский язык

\usepackage[T2A]{fontenc}			% кодировка
\usepackage[utf8]{inputenc}			% кодировка исходного текста
\usepackage[english,russian]{babel}	% локализация и переносы


% Математика
\usepackage{amsmath,amsfonts,amssymb,amsthm,mathtools} 


\usepackage{wasysym}

\usepackage{hyperref} % гиперссылки

%Заговолок
\author{Литовченко Даниил}
\title{Команды для создания текстового документ в \LaTeX{}}
\date{16.12.2020}


\begin{document} % начало документа

\maketitle
\newpage

\textbackslash documentclass[]\{класс\}\\
Задаёт общий стиль оформления («класс документа»)\\
Классы:\\
\begin{itemize}
\item article
\item amsart (для оформления статей)
\item report (похоже на article и book) 
\item и другие
\end{itemize}
В квадратных скобках прописывают формат бумаги и размер шрифта.\\
Например:\\
\textbackslash documentclass[a4paper,12pt]\{article\}\\
После подключаются необходимые пакеты с помощью команды \textbackslash usepackage[]\{\}\\
Например, подключение русского языка:\\
\textbackslash usepackage[cp1251]\{inputenc\}\\
\textbackslash usepackage[russian]\{babel\}\\

Сам документ находится между командами
\textbackslash begin\{document\}\\
\textbackslash end\{document\}\\
\textbackslash parindent        обозначает в \TeX{} величину абзацного отступа.
Например:\\
\textbackslash parindent=2cm\\

\textbackslash subsection и \textbackslash subsection* создают разделы документа с нумерацией и без нумерации соответственно.

\end{document}