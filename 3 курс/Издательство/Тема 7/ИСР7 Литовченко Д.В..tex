% "Лабораторная работа 3"

\documentclass[a4paper,12pt]{article} % тип документа

% report, book

%  Русский язык

\usepackage[T2A]{fontenc}			% кодировка
\usepackage[utf8]{inputenc}			% кодировка исходного текста
\usepackage[english,russian]{babel}	% локализация и переносы


% Математика
\usepackage{amsmath,amsfonts,amssymb,amsthm,mathtools} 


\usepackage{wasysym}

\usepackage{hyperref} % гиперссылки

%Заговолок
\author{Царулкова Анастасия}
\title{Работа с текстом в \LaTeX{}}
\date{21.10.2020}


\begin{document} % начало документа

\maketitle
\newpage
\section{Вопросы к пособию с.49-63}
\begin{enumerate}
\item Как настроить поддержку кириллицы?\\
В преамбуле нужно написать команды\\
$\backslash$usepackage[T2A]\{fontenc\}\\
usepackage[utf8]\{inputenc\}\\	
$\backslash$usepackage[russian]\{babel\}
\item Как разбить документ на части без изменения нумерации разделов?\\
Использовать команду $\backslash$part{}
\item Как создать титульный лист?
Нужно прописать команду $\backslash$maketitle
Содержимое титула определяется командами, которые прописываются заранее:\\
$\backslash$title\{...\}, $\backslash$author\{...\}, $\backslash$date\{...\}
\item Как создать перекрестную ссылку?\\
Для этого используются команды\\
$\backslash$label\{метка\}, $\backslash$ref\{метка\}, $\backslash$pageref\{метка\},\\
Метка -- это выбранный пользователем идентифиатор. \LaTeX{} заменяет $\backslash$ref номером раздела, подраздела, иллюстрации и др.
\item Как создать сноску?\\
Сноска создатется с помощью команды $\backslash$footnote\{текст сноски\}. Сноски всегда должны помещаться после слова или предложения, к которым они относятся.
\item Как создается окружение?\\
Окружение создаётся с помощбю команд \\$\backslash$begin\{\}\\
Текст\\
$\backslash$end\{\}
\item Как создать таблицу?\\
Для этого создается окружение $\backslash$begin\{tabular\}[позиция]\{спецификация\}\\
Позиция определяет вертикальное положение всего табличного окуржения: t, b, c -- выравнивание по верхнему, нижнему краю и центру.\\ Спецификация определяет формат таблицы: l, r, c -- выравнивают столбец текста по левому, правому краю и центру соответственно.\\
Знак \& создает переход к следующему столбцу, $\backslash$ $\backslash$ создает новую строку, а $\backslash$hline вставляет горизонтальную линию.
\item Как оформить цитату?\\
Это можно сделать с помощью окружения $\backslash$begin\{quote\}\\
Также есть похожие окружения:\\
quotation -- для длинных цитат\\
verse -- для стихов
\item Как создать перечисление?\\
Простой список: $\backslash$begin\{itemize\}\\
Нумерованный список: $\backslash$begin\{enumerate\}
\item Как создать описание?
Описание создается с помощью окружения $\backslash$begin\{description\}
\end{enumerate}

\end{document} % конец документа