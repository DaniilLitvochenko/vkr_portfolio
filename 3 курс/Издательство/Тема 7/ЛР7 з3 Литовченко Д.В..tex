% "Лабораторная работа 3"

\documentclass[a4paper,12pt]{article} % тип документа

% report, book

%  Русский язык

\usepackage[T2A]{fontenc}			% кодировка
\usepackage[utf8]{inputenc}			% кодировка исходного текста
\usepackage[english,russian]{babel}	% локализация и переносы


% Математика
\usepackage{amsmath,amsfonts,amssymb,amsthm,mathtools} 


\usepackage{wasysym}

\usepackage{hyperref} % гиперссылки

%Заговолок
\author{Литовченко Даниил}
\title{Работа с текстом в \LaTeX{}}
\date{16.12.2020}


\begin{document} % начало документа

\maketitle
\newpage
\section{Задание 3}
\begin{enumerate}
\item Вычислить значения функции $y(x)$ для каждого $x$. Коэффициенты $t, k, s$ являются константами и вводятся с клавиатуры. Значение $x$ находится в интервале $[-25;15]$ и изменяется с шагом 1.
$$y=t\cdot x^3+k\cdot x+s$$
\item Изменяя значение переменной $k$ (начальное значение $k=1$, шаг 1), найдите при каком $k$ значение функции $y(k)$ превысит 1200.
$$y=2^{k+2}-5$$
\item В данной функции $w,n,c$ - константы, $x$ - вводится с клавиатуры. Найти значение функции.
\begin{equation*}
	y=
	\begin{cases}
		w^2,\text{при } x\geq1.5\\
		n\cdot x+9, \text{при } x\in (-12:1.5)\\
		c - x, \text{при } x\leq -12
	\end{cases}
\end{equation*}
\end{enumerate}

\end{document} % конец документа