% "Лабораторная работа 3"

\documentclass[a4paper,12pt]{article} % тип документа

% report, book

%  Русский язык

\usepackage[T2A]{fontenc}			% кодировка
\usepackage[utf8]{inputenc}			% кодировка исходного текста
\usepackage[english,russian]{babel}	% локализация и переносы


% Математика
\usepackage{amsmath,amsfonts,amssymb,amsthm,mathtools} 


\usepackage{wasysym}

\usepackage{hyperref} % гиперссылки

%Заговолок
\author{Царулкова Анастасия}
\title{Работа с текстом в \LaTeX{}}
\date{21.10.2020}


\begin{document} % начало документа

\maketitle
\newpage
\section{Формулы}
В тексте есть формула $a+b$ суммы двух чисел.
Есть формула, которая записана с новой строки $$a^2+b^2=c^2$$ в нашем тексте
\begin{equation} \label {pif}
a^2+b^2=c^2
\end{equation}

Теорема пифагора \eqref{pif} известна из школьного курса математики

\section{Индексы}
$m_1$ \\
$m_{12}$ \\
$c^2$ \\
$c^{22}$

\section{Стандартные функции}
$\sin(x)$ \\
$\sin x$ \\
$\arctg x= \sqrt{3} $\\
$\arctg x=$ $\sqrt[3]{5}$ \\


\begin{equation}
\arctg x=\sqrt{3}
\end{equation}
\\
\begin{equation}
\arctg x=\sqrt{5}{3}
\end{equation}
$\log_{x-1}{x^2-3x-4} \geqslant 2$ \\
$\lg 10 - \ln e$

\section{Функции}
Формула суммы $\sum_{i=1}^{n}a_i+b_i$ некоторых значений \\
\[ \sum_{i=1}^{n}a_i+b_i \]
$ I=\int r^dm интегрирование по m $ \\
\[I=\int r^2dm\]\\
\[I=\int\limits_{0}^{1} r^2dm\]\\

\section{Знаки}

$ 3\times2=2\cdot 3 $ \\
$1 \dots n $\\

$(\frac{2}{4}= \frac{1}{2})$
\[\underbrace{1+2+3+\dots+n} = N \]

\end{document} % конец документа