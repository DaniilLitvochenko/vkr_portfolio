% "Лабораторная работа 2"

\documentclass[a4paper,12pt]{article} % тип документа

% report, book

%  Русский язык

\usepackage[T2A]{fontenc}			% кодировка
\usepackage[utf8]{inputenc}			% кодировка исходного текста
\usepackage[english,russian]{babel}	% локализация и переносы


% Математика
\usepackage{amsmath,amsfonts,amssymb,amsthm,mathtools} 


\usepackage{wasysym}

\usepackage{hyperref} % гиперссылки

%Заговолок
\author{Литовченко Даниил}
\title{Работа с текстом в \LaTeX{}}
\date{16.12.2020}


\begin{document} % начало документа

\maketitle
\newpage
\section{Для чего предназначена издательская система \LaTeX{}?}
\begin{flushright}
\LaTeX{} - это настольная издательская система. Она позволяет создавать множество типов документов: от  одностраничных писем до создания многотомных фолиантов.  \LaTeX{} поддерживает все существующие типы компьютеров. \LaTeX{} упрощает работу с текстом, позволяя сосредоточить внимание на его содержании.
\end{flushright}

\section{В каких случаях рационально её использовать?}
\begin{center}
Для подготовки печатных документов, создания научных статей. Особенно удобно её использовать в документах, в которых много математических формул.
\end{center}

\section{Какие преимущества имеет работа в этот системе?}
\begin{flushleft}
\begin{itemize}
\item Готовые профессионально выполненные макеты, делающие документы действительно выглядящими <<как изданные>>.
\item Удобная верстка математических формул.
\item Низкий порог вхождения: нужно выучить лишь несколько понятных команд, задающих логическую структуру документа. Зачастую не нужно возиться собственно с макетом документа.
\item Легко изготавливаются даже сложные структуры: примечания, оглавления, библиографии\footnote{Научное, систематизированное по какому-л. признаку перечисление и описание книг и других изданий.} и прочее.
\item Существуют свободно распространяемые дополнительные пакеты для многих типографских задач, не поддерживаемых напрямую базовым \LaTeX{}. Например, наличествуют пакеты для включения POSTSCRIPT графики или для верстки библиографий в точном соответствии с конкретными стандартами.
\item \LaTeX{} поощряет авторов писать хорошо структурированные документы, так как именно так \LaTeX{} и работает -- путем спецификации структуры.
\item \TeX{}, форматирующее сердце \LaTeXe{}, чрезвычайно мобилен и свободно доступен. Поэтому система работает практически на всех существующих платформах.
\end{itemize}
\end{flushleft}

\section{Какие сложности могут возникнуть при работе в этот системе?}
\begin{enumerate}
\item Хотя готовые макеты имеют множество настраиваемых параметров, создание {\Large полностью} нового макета документа не очень просто и занимает много времени.
\item Очень сложно писать неструктурированные и неорганизованные документы.
\item Создание новых стилей оформление - дело сложное и под силу лишь профессионалам. Обычный пользователь, как правило, с такой задачей не справится.
\end{enumerate}

\section{Какие недостатки отмечают пользователи при работе с этой системой?}
\begin{itemize}
\item Не WISIWIG. Требуется владеть навыками работы в редакторе текста.
\item Требует знания элементарных основ полиграфии.
\end{itemize}

\section{Источники}
\begin{itemize}
\item \href{https://astronu.jinr.ru/wiki/images/b/bf/KotelnikovChebotaev.pdf}{ссылка}
\item \href{http://tex.imm.uran.ru/tex/2e/lshort2e/node8.html}{ссылка}
\item \href{https://www.opennet.ru/docs/RUS/linux_base/node381.html}{ссылка}
\item \href{https://kartaslov.ru/значение-слова/библиография}{ссылка}
\end{itemize}

\section{Вопросы и ответы}
\begin{enumerate}
\item Кто создал \TeX{}?

Дональд Кнут
\item Как начать новый абзац?

Вставить пустую строку.
\item Как использовать спецсимволы в тексте?

Нужно экранировать их с помощью символа "$\backslash$". Сам символ "$\backslash$" добавляется в текст в с помощью комнады \$ $\backslash$backslash \$ .

\item Какие есть способы создания комментария в \LaTeX{}

Либо с помощью знака \% в начале строки, либо можно создать многострочный коментарий с помощью окружения comment. Пример: $\backslash$begin\{comment\} \\
Комментарий \\
$\backslash$end\{comment\}
\item Как подключать пакеты?

Это можно сделать с помощью команды $\backslash$usepackage\{\}
\item Как настроить размер шрифта в документе по умолчанию?
Это настраивают с помощью команды $\backslash$documentclass[]\{\}

\item Какие существуют классы документов в \LaTeX{}

Классы документов:\\
article -- для статей, научных журналов, презентация, которотких отчётов, локументаций и др.\\
report -- для более длинных отчётов, содержащих несколько глав, небольших книжек, диссертаций.
book -- для настоящих книг\\
slides -- для слайдов. Использует больште буквы без засечек.

\item Какие предопределенные стили страниц существуют в \LaTeX{}

plain -- печатает номера страниц внизу страницы в середине нижнего колонотитула. Этот стиль установлен по умолчанию.\\
headings -- печатает название текущей главы и номер страницы, а нижний колонотитул остается пустым.\\
empty делает верхние и нижние колонотитулы пустыми.
\item Как напечатать знак градуса?

Это можно сделать с помощью команды  \^\{$\backslash$circ\}
\item Как звучит команда для создания многоточия?

\end{enumerate}


\end{document} % конец документа